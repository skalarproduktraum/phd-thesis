% Table of contents formatting
\renewcommand{\contentsname}{Contents}
\setcounter{secnumdepth}{2}
\setcounter{tocdepth}{1}
 

% Headers and page numbering 
\usepackage{fancyhdr}
\usepackage{fontspec}
\usepackage{nextpage,epigraph}
\usepackage{tex/epipart}
\usepackage{titlesec}
\usepackage{appendix}
\usepackage{pdfpages}
\usepackage{qrcode}
\usepackage{xpatch}



\pagestyle{plain}
 
% Fonts and typesetting
\usepackage[usenames,svgnames]{xcolor}
\usepackage{colortbl}
\newfontfamily{\headingfont}[Path=fonts/, 
    Extension = .ttf, 
    UprightFont = *-Regular,
    BoldFont = *-Bold]{RobotoSlab}

\setmonofont[Path=fonts/, 
    Extension = .otf, 
    UprightFont = *-Regular,
    BoldFont = *-Bold, 
    ItalicFont = *-It, 
    BoldItalicFont = *-BoldIt]{SourceCodePro}

\setmainfont[Path=fonts/, 
    Extension = .ttf, 
    UprightFont = *-Regular, 
    BoldFont = *-Bold, 
    ItalicFont = *-Italic, 
    BoldItalicFont = *-BoldItalic]{EBGaramond}

\newfontfamily{\tocpartfont}[Path=fonts/, 
    Extension = .ttf, 
    UprightFont = *-Regular, 
    BoldFont = *-Bold, 
    ItalicFont = *-Italic, 
    BoldItalicFont = *-BoldItalic,
    LetterSpace = 20]{EBGaramond}


\usepackage{siunitx}

% Adjust spacing between lines to 1.5
\usepackage{setspace}
\onehalfspacing
\raggedbottom

% Chapter styling
\usepackage[grey]{quotchap}
\makeatletter 
\renewcommand*{\chapnumfont}{%
  \usefont{T1}{\@defaultcnfont}{b}{n}\fontsize{80}{100}\selectfont% Default: 100/130
  \color{chaptergrey}%
}
\makeatother

\newcommand{\yellowbox}[1]{\colorbox{yellow}{#1}}
\titleformat{\chapter}[display]
  {\LARGE\headingfont\begin{minipage}{\linewidth}}
  {\chaptertitlename\ \thechapter:}
  {0pt}
  {\raggedright\Huge\yellowbox}
  [\end{minipage}]
\titleformat*{\section}{\LARGE\headingfont\raggedright}
\titleformat*{\subsection}{\Large\headingfont\raggedright}
\makeatletter
\titleformat{\part}[display]
  {\Huge\headingfont\filright}
  {\partname~\thepart:}
  {0pt}
  {\thispagestyle{epigraph}\Huge\yellowbox}
\makeatother
\setlength\epigraphwidth{0.6\textwidth}

\makeatletter
\xpatchcmd\epigraphhead
 {\let\@evenfoot}
 {\let\@oddfoot\@empty\let\@evenfoot}
 {}{}
\makeatother

% Set colour of links to black so that they don't show up when printed
\PassOptionsToPackage{unicode=true,backref=page}{hyperref} % options for packages loaded elsewhere
\usepackage{hyperref}

% letter spacing — workaround needed by tufte-book package
% Set up the spacing using fontspec features
\renewcommand\allcapsspacing[1]{{\addfontfeature{LetterSpace=15}#1}}
\renewcommand\smallcapsspacing[1]{{\addfontfeature{LetterSpace=10}#1}}

% Tables
\usepackage{booktabs}
\usepackage{threeparttable}
\usepackage{array}
\usepackage{unicode-math}
\usepackage{algorithm2e}
\usepackage{mathtools}
\usepackage{menukeys}
\DeclarePairedDelimiter{\ceil}{\lceil}{\rceil}
\DeclarePairedDelimiter\floor{\lfloor}{\rfloor}

\newcolumntype{x}[1]{%
>{\centering\arraybackslash}m{#1}}%

% TODO command
\newcommand{\TODO}[1]{\colorbox{OrangeRed}{\textcolor{White}{\textbf{\emph{TODO:} #1}}}}

% Epigraph
\newcommand{\myepigraph}[2]{\epigraphhead[450]{\raggedright#1\par\hspace*{\fill}\textsc{---#2}}}

\usepackage{hyperxmp}[2020/03/01]
\usepackage{embedfile}[2020/04/01]

%\usepackage{bookmark}

\hypersetup{
    bookmarksdepth=3,
    bookmarksopen=true,
    bookmarksopenlevel=2,
    bookmarksnumbered=true,
    colorlinks=true,
    linkcolor={MidnightBlue},
    citecolor={MidnightBlue},
    urlcolor={MidnightBlue},
    pagebackref=true,
    backref=page,
    pdflang=en,
    pdfapart=2, %set to 1 for PDF/A-1
    pdfaconformance=B,
    pdftitle={A Modular and Open-Source Framework for Virtual Reality Visualisation and Interaction in Bioimaging},
    pdfauthor={Ulrik Günther},
    pdfcreator={Ulrik Günther},
    pdfsubject={PhD thesis, visualisation in bioimaging},
    pdfcopyright={Copyright (c) 2018-2020, Ulrik Günther},
    pdfdate={2020-07-23},
    pdfkeywords={Visualisation,
               virtual reality,
               systems biology,
               human-computer interaction,
               eye tracking,
               augmented reality,
               cross reality,
               Kotlin,
               Java,
               ImageJ,
               Cross-platform,
               open-source,
               software,
               bioimaging}
}

%Create an OutputIntent in order to correctly specify colours
\immediate\pdfobj stream attr{/N 3} file{sRGB.icc}
\pdfcatalog{%
  /OutputIntents [
    <<
      /Type /OutputIntent
      /S /GTS_PDFA2
      /DestOutputProfile \the\pdflastobj\space 0 R
      /OutputConditionIdentifier (sRGB)
      /Info (sRGB)
    >>
  ]
}

\usepackage{etoolbox}
\makeatletter
\patchcmd{\BR@backref}{\newblock}{\newblock(page~}{}{}
\patchcmd{\BR@backref}{\par}{)\par}{}{}
\makeatother

\makeatletter
\patchcmd{\l@section}{#1}{\MakeTextUppercase{#1}}{}{}
\makeatother

% \renewcommand{\citep}[1]{\cite{#1}}
% \let\citep\cite

% smaller fonts for algorithms
\SetAlFnt{\small}

% look in the figures path by default
\graphicspath{{../figures/}{figures/}{./}{../figures/external/}{figures/external/}}

\newlength\publishedboxwidth
\setlength{\publishedboxwidth}{\dimexpr\textwidth-2\fboxsep}
\usepackage[skins]{tcolorbox}
\definecolor{publishedbg}{RGB}{240,240,240}
\tcbset{commonstyle/.style={boxrule=0pt,sharp corners,enhanced jigsaw,nobeforeafter,boxsep=0pt,left=\fboxsep,right=\fboxsep}}
\newtcolorbox{publishedin}{commonstyle,colback=publishedbg}
\newcommand{\alreadypublished}[2]{
\begin{publishedin}
    #1
    \vskip 0.3em
    #2
\end{publishedin}
}



% DOI linking with hyperref
\renewcommand{\doi}[1]{\href{http://doi.org/#1}{DOI: \nolinkurl{#1}}}


% Code Listings
\usepackage{multicol}
\usepackage{listings}
\usepackage{tex/glslListings}
\usepackage{tex/kotlinListings}

\lstset{
    basicstyle=\scriptsize\ttfamily\linespread{0.7},
    backgroundcolor=\color{white},
    breaklines=true,
	frame=none,
	tabsize=4,
	columns=fixed,
	showstringspaces=false,
	showtabs=false,
	keepspaces,
    numbers=left,
	keywordstyle=\color{blue},
	commentstyle=\color{OliveGreen},
    numberstyle=\tiny\color{Gray},
    numbersep=5pt,
    captionpos=b,
    abovecaptionskip=1.2em
	}


% YAML in listings
\newcommand\YAMLcolonstyle{\color{Gray}\ttfamily}
\newcommand\YAMLkeystyle{\color{MidnightBlue}\ttfamily\scriptsize\linespread{0.7}}
\newcommand\YAMLvaluestyle{\color{Green}\ttfamily}

\makeatletter

% here is a macro expanding to the name of the language
% (handy if you decide to change it further down the road)
\newcommand\language@yaml{yaml}

\expandafter\expandafter\expandafter\lstdefinelanguage
\expandafter{\language@yaml}
{
  keywords={true,false,null,y,n},
  keywordstyle=\color{darkgray}\ttfamily,
  basicstyle=\YAMLkeystyle,                                 % assuming a key comes first
  sensitive=false,
  comment=[l]{\#},
  morecomment=[s]{/*}{*/},
  commentstyle=\color{Gray}\ttfamily,
  stringstyle=\YAMLvaluestyle\ttfamily,
  moredelim=[l][\color{orange}]{\&},
  moredelim=[l][\color{magenta}]{*},
  moredelim=**[il][\YAMLcolonstyle{:}\YAMLvaluestyle]{:},   % switch to value style at :
  morestring=[b]',
  morestring=[b]",
  literate =    {---}{{\ProcessThreeDashes}}3
                {>}{{\textcolor{red}\textgreater}}1     
                {|}{{\textcolor{red}\textbar}}1 
                {\ -\ }{{\ttfamily\ -\ }}3,
}

% switch to key style at EOL
\lst@AddToHook{EveryLine}{\ifx\lst@language\language@yaml\YAMLkeystyle\fi}
\makeatother

\newcommand\ProcessThreeDashes{\llap{\color{cyan}\ttfamily-{-}-}}
 
% needs caption package, collides with tufte-book
%\DeclareCaptionFormat{lstlisting}{\normalfont\small #1#2#3}
%\captionsetup[lstlisting]{font=small,labelfont=small}
%
\usepackage[capitalise,noabbrev]{cleveref}


% JavaScript
\definecolor{lightgray}{rgb}{.9,.9,.9}
\definecolor{darkgray}{rgb}{.4,.4,.4}
\definecolor{purple}{rgb}{0.65, 0.12, 0.82}
\lstdefinelanguage{JavaScript}{
  keywords={break, case, catch, continue, debugger, default, delete, do, else, false, finally, for, function, if, in, instanceof, new, null, return, switch, this, throw, true, try, typeof, var, void, while, with},
  morecomment=[l]{//},
  morecomment=[s]{/*}{*/},
  morestring=[b]',
  morestring=[b]",
  ndkeywords={class, export, boolean, throw, implements, import, this},
  sensitive=true
}

% Use cleveref to actually show numbers in front the the section names
\renewcommand{\hyperlink}[2]{\cref{#1}, \emph{#2}}

\makeatletter
\renewcommand{\@tufte@lof@line}[2]{%
  % #1 is the figure/table number and its caption text
  % #2 is the page number on which the figure/table appears
  \leftskip 0.0em
  \rightskip 0em
  \parfillskip 0em plus 1fil
  \parindent 0.0em
  \@afterindenttrue
  \interlinepenalty\@M
  \leavevmode
  \@tempdima 2.0em
  \advance\leftskip\@tempdima
  \null\nobreak\hskip -\leftskip
  {#1}\nobreak\nobreak\hspace*{0em}\hfill #2%
  \par%
}
\makeatother


%% pandoc-tablenos: required package
%\usepackage{caption}

%% pandoc-tablenos: environment to disable table caption prefixes
\makeatletter
\newcounter{tableno}
\newenvironment{tablenos:no-prefix-table-caption}{
}{
}
\makeatother
